\documentclass{article}
\usepackage{graphicx, subfig, fancyhdr, amsmath, amssymb, amsthm, url, geometry, listings, xcolor}
\usepackage[utf8]{inputenc}
\usepackage[margin=1in]{geometry}
% Fix Unicode character issues
\DeclareUnicodeCharacter{2212}{-}
% Code listing style
\lstset{
  language=C,
  basicstyle=\ttfamily\small,
  numbers=left,
  numberstyle=\tiny,
  stepnumber=1,
  frame=single,
  backgroundcolor=\color{gray!10},
  keywordstyle=\color{blue}\bfseries,
  commentstyle=\color{green!50!black},
  stringstyle=\color{red},
  breaklines=true,
  breakatwhitespace=true,
  showstringspaces=false,
  tabsize=4
}
% Define authors
\newcommand{\FirstAuthor}{Mohammad Parsa Dini - Std ID: 400101204}
% \newcommand{\SecondAuthor}{[Second Author Name - Std ID: XXX]} % Placeholder
\newcommand{\exerciseset}{Control LAB - HW2}
% Fancy header and footer
\fancypagestyle{plain}{\fancyhf{}\fancyfoot[RO,LE]{\sffamily\bfseries\thepage}}
\pagestyle{fancy}
\fancyhf{}
\fancyhead[RO,LE]{\sffamily\bfseries\large Sharif University of Technology}
\fancyhead[LO,RE]{\sffamily\bfseries\large EE 25-403: Control LAB}
\fancyfoot[LO,RE]{\sffamily\bfseries\large Control LAB HW2}
\fancyfoot[RO,LE]{\sffamily\bfseries\thepage}
\renewcommand{\headrulewidth}{1pt}
\renewcommand{\footrulewidth}{1pt}
\graphicspath{{figures/}}
\title{
  \vspace{-2em}
  \includegraphics[width=3cm]{logo.png} \\
  \vspace{0.5em}
  Control LAB \\
  \exerciseset
}
\author{\FirstAuthor \and \SecondAuthor}
\date{May 2025}
% Load hyperref last
\usepackage{hyperref}

\begin{document}
\maketitle

\section{System Identification and Control}
\begin{enumerate}
  \item Linearize the system around the operating point $(u_0,y_0)\approx(0.2,0.5)$.
  \begin{equation*}
    \dot{y} = -\frac{y}{5} + \frac{4.5}{20}\sqrt{u} = -0.2 y + 0.225 \sqrt{u}
  \end{equation*}
  Since by Taylor representation we have $\sqrt{u}|_{u_0=0.2} \approx \sqrt{u_0} + \frac{u-u_0}{2\sqrt{u_0}} \approx 0.447 + 1.118(u-0.2) = 1.118u + 0.224$. Thus around $u=u_0=0.2$:
  \begin{equation*}
    \dot{y} = -0.2y + 0.225(1.118u + 0.224) = -0.2y + 0.252u + 0.0504
  \end{equation*}
  which suggests around the operating point: $\dot{y} \approx 0.002$. Taking the Laplace transform of $\dot{y} \approx -0.2y + 0.25u$ results in:
  \begin{equation*}
    sY(s) = -0.2Y(s) + 0.25 U(s) \implies H(s) = \frac{Y(s)}{U(s)} = \frac{0.25}{s + 0.2}
  \end{equation*}

  \item Attempt to directly derive a first-order linear model for the nonlinear system shown in the figure. In order to do so, apply a small step change around the operating point $(u_0,y_0)=(0.2,0.5)$ and determine the system's gain and time constant. \\
  \begin{equation*}
    \dot{y} = -\frac{y}{5} + \frac{4.5}{20}\sqrt{u} \implies \dot{y} = -0.1 + \frac{4.5}{20}\sqrt{0.2} \approx 0.0006
  \end{equation*}
  Let $u_1 = u_0 + \Delta u = 0.2 + 0.02 = 0.22 \implies 0.0006 = -0.2y_1 + \frac{4.5}{20}\sqrt{0.22} \approx 0.500115$, which suggests $\Delta y = 0.000115 \implies \text{gain} = \frac{\Delta y}{\Delta u} = \frac{0.000115}{0.02} \approx 0.00575$. The Simulink result is depicted below:
  \begin{figure}[h!]
    \centering
    \includegraphics[width=0.8\linewidth]{Screenshot 2025-07-07 111042.png}
    \caption{Simulink result for step response around operating point.}
    \label{fig:fig1}
  \end{figure}
  We can deduce that $K = \frac{0.562-0.5}{0.05} \approx 1.24$ (from Simulink). However, the $K=1.08$ from the .mlx file is more reliable.
  \begin{figure}[h!]
    \centering
    \includegraphics[width=0.8\linewidth]{Screenshot 2025-07-07 112245.png}
    \caption{Simulink output for gain calculation.}
    \label{fig:fig2}
  \end{figure}
  $y_2 = 0.5 + 0.63 (0.562 - 0.5) \approx 0.539 \implies t_2 = 34.7 \implies \tau = t_2 - 30 = 4.7$ (from Simulink).
  \begin{figure}[h!]
    \centering
    \includegraphics[width=0.8\linewidth]{Screenshot 2025-07-07 112713.png}
    \caption{Simulink output for time constant calculation.}
    \label{fig:fig3}
  \end{figure}

  \item Compare the linear model derived in parts 1 and 2 with the actual system by applying a step change around the operating point $(u_0,y_0)=(0.2,0.5)$. Plot the output of models and the real systems on the same axes. \\
  The results are shown in the .mlx file attached to the GitHub repository and this report. The result from Simulink:
  \begin{figure}[h!]
    \centering
    \includegraphics[width=0.8\linewidth]{Screenshot 2025-07-07 113152.png}
    \caption{Comparison of linear model and actual system response.}
    \label{fig:fig4}
  \end{figure}
  As you can see, $G_a(s)$ performed better in MATLAB and Simulink.

  \item Design a PI controller to raise the output from 0.5 to 0.6 while satisfying the following requirements: \\
  - Zero steady-state error in response to a unit step input. \\
  - Overshoot less than 10\%. \\
  - Settling time less than 10 seconds. \\
  The PI controller is given by $u(t) = K_p (\beta r(t) - y(t)) + K_i \int_{0}^{t} (r(t) - y(t)) \, dt$. Since $\beta=0$, we have $U(s) = -K_p Y(s) + K_i \frac{R(s)-Y(s)}{s}$. Using $Y(s) = G(s)U(s)$, the closed-loop transfer function is:
  \begin{equation*}
    \frac{Y(s)}{R(s)} = \frac{G(s) K_i / s}{1 + G(s) (K_p + K_i / s)} = \frac{K_i G(s)}{s + G(s) (K_p s + K_i)}
  \end{equation*}
  Let $G(s) = \frac{K}{\tau s + 1}$, so:
  \begin{equation*}
    \frac{Y(s)}{R(s)} = \frac{\frac{K K_i}{\tau}}{s^2 + \frac{K K_p + 1}{\tau} s + \frac{K K_i}{\tau}}
  \end{equation*}
  The first criterion (zero steady-state error) is met since:
  \begin{equation*}
    \lim_{s \to 0} \frac{Y(s)}{R(s)} = 1 \implies e_{ss} = \lim_{t \to \infty} e(t) = 0
  \end{equation*}
  For the second criterion (overshoot):
  \begin{equation*}
    \text{Overshoot} = e^{-\pi \zeta / \sqrt{1-\zeta^2}} \leq 0.1 \implies \zeta \geq \sqrt{\frac{\ln(0.1)^2}{\ln(0.1)^2 + \pi^2}} \approx 0.591
  \end{equation*}
  Choosing $\zeta = 0.65$ satisfies $\zeta \geq 0.591$. For the third criterion (settling time):
  \begin{equation*}
    T_s = \frac{4}{\zeta \omega_n} \leq 10 \implies 2 \zeta \omega_n \geq \frac{8}{10} \implies \frac{K K_p + 1}{\tau} \geq 0.8
  \end{equation*}
  Using $G(s) = \frac{0.25}{s + 0.2} = \frac{1.25}{5s + 1}$, we have $K = 1.25$, $\tau = 5$. Thus:
  \begin{equation*}
    K_p \geq \frac{\frac{8 \cdot 5}{10} - 1}{1.25} = 2.4
  \end{equation*}
  Choosing $K_p = 3.4$ satisfies the condition.
\end{enumerate}

\section*{Conclusion}
This lab demonstrated the process of system identification and control design for a nonlinear system. By linearizing the system around an operating point, deriving a first-order model, and comparing it with the actual system, we validated the linear approximation. The PI controller design successfully met the requirements of zero steady-state error, overshoot less than 10\%, and settling time less than 10 seconds, as verified through Simulink simulations.

\end{document}
