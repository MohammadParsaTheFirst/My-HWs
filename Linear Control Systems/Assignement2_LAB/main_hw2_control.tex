\documentclass{article}
\usepackage[utf8]{inputenc}
\usepackage{graphicx}
\usepackage{subcaption}
\usepackage{amsmath, amssymb, amsthm}
\usepackage{url, geometry, xcolor}
\usepackage{fancyhdr}
\usepackage{hyperref}

% Fix Unicode character issues
\DeclareUnicodeCharacter{2212}{-}

% Page geometry
\usepackage[margin=1in]{geometry}

% Define authors
\newcommand{\FirstAuthor}{Mohammad Parsa Dini - Std ID: 400101204}
% \newcommand{\SecondAuthor}{[Second Author Name - Std ID: XXX]} % Uncomment and define if needed
\newcommand{\exerciseset}{Control LAB - HW2}

% Fancy header and footer
\fancypagestyle{plain}{\fancyhf{}\fancyfoot[RO,LE]{\sffamily\bfseries\thepage}}
\pagestyle{fancy}
\fancyhf{}
\fancyhead[RO,LE]{\sffamily\bfseries\large Sharif University of Technology}
\fancyhead[LO,RE]{\sffamily\bfseries\large EE 25-403: Control LAB}
\fancyfoot[LO,RE]{\sffamily\bfseries\large Control LAB HW2}
\fancyfoot[RO,LE]{\sffamily\bfseries\thepage}
\renewcommand{\headrulewidth}{1pt}
\renewcommand{\footrulewidth}{1pt}

\graphicspath{{figures/}}

\title{
  \vspace{-2em}
  \includegraphics[width=3cm]{logo.png} \\
  \vspace{0.5em}
  Control LAB \\
  \exerciseset
}
\author{\FirstAuthor} % Add \and \SecondAuthor if defined
\date{May 2025}

\begin{document}
\maketitle

\section{System Identification and Control}
\begin{enumerate}
  \item Linearize the system around the operating point $(u_0,y_0)\approx(0.2,0.5)$.
  \begin{equation*}
    \dot{y} = -\frac{y}{5} + \frac{4.5}{20}\sqrt{u} = -0.2 y + 0.225 \sqrt{u}
  \end{equation*}
  Using Taylor expansion, $\sqrt{u}|_{u_0=0.2} \approx \sqrt{u_0} + \frac{u-u_0}{2\sqrt{u_0}} \approx 0.447 + 1.118(u-0.2) = 1.118u + 0.224$. Thus, around $u=u_0=0.2$:
  \begin{equation*}
    \dot{y} = -0.2y + 0.225(1.118u + 0.224) = -0.2y + 0.252u + 0.0504
  \end{equation*}
  which suggests around the operating point: $\dot{y} \approx 0.002$. Taking the Laplace transform of $\dot{y} \approx -0.2y + 0.25u$ results in:
  \begin{equation*}
    sY(s) = -0.2Y(s) + 0.25 U(s) \implies H(s) = \frac{Y(s)}{U(s)} = \frac{0.25}{s + 0.2}
  \end{equation*}

  \item Attempt to directly derive a first-order linear model for the nonlinear system shown in the figure. Apply a small step change around the operating point $(u_0,y_0)=(0.2,0.5)$ and determine the system's gain and time constant.
  \begin{equation*}
    \dot{y} = -\frac{y}{5} + \frac{4.5}{20}\sqrt{u} \implies \dot{y} = -0.1 + \frac{4.5}{20}\sqrt{0.2} \approx 0.0006
  \end{equation*}
  Let $u_1 = u_0 + \Delta u = 0.2 + 0.02 = 0.22 \implies 0.0006 = -0.2y_1 + \frac{4.5}{20}\sqrt{0.22} \approx -0.2y_1 + 0.1056$, which gives $y_1 \approx 0.5253$. Thus, $\Delta y = 0.5253 - 0.5 = 0.0253 \implies \text{gain} = \frac{\Delta y}{\Delta u} = \frac{0.0253}{0.02} \approx 1.265$. The Simulink result is depicted below:
  \begin{figure}[h!]
    \centering
    \includegraphics[width=0.8\linewidth]{Screenshot 2025-07-07 111042.png}
    \caption{Simulink result for step response around operating point.}
    \label{fig:step_response}
  \end{figure}
  From Simulink, $K = \frac{0.562-0.5}{0.05} \approx 1.24$. However, the $K=1.08$ from the .mlx file (available in the GitHub repository) is more reliable.
  \begin{figure}[h!]
    \centering
    \includegraphics[width=0.8\linewidth]{Screenshot 2025-07-07 112245.png}
    \caption{Simulink output for gain calculation.}
    \label{fig:gain_calc}
  \end{figure}
  For the time constant, $y_2 = 0.5 + 0.63 (0.562 - 0.5) \approx 0.539 \implies t_2 = 34.7 \implies \tau = t_2 - 30 = 4.7$ (from Simulink).
  \begin{figure}[h!]
    \centering
    \includegraphics[width=0.8\linewidth]{Screenshot 2025-07-07 112713.png}
    \caption{Simulink output for time constant calculation.}
    \label{fig:time_constant}
  \end{figure}

  \item Compare the linear model derived in parts 1 and 2 with the actual system by applying a step change around the operating point $(u_0,y_0)=(0.2,0.5)$. Plot the output of models and the real systems on the same axes.
  The results are shown in the .mlx file attached to the GitHub repository and this report. The Simulink result:
  \begin{figure}[h!]
    \centering
    \includegraphics[width=0.8\linewidth]{Screenshot 2025-07-07 113152.png}
    \caption{Comparison of linear model and actual system response.}
    \label{fig:model_comparison}
  \end{figure}
  As shown, $G_a(s)$ performed better in MATLAB and Simulink.

  \item Design a PI controller to raise the output from 0.5 to 0.6 while satisfying:
  \begin{itemize}
    \item Zero steady-state error for a unit step input.
    \item Overshoot less than 10\%.
    \item Settling time less than 10 seconds.
  \end{itemize}
  The PI controller is $u(t) = K_p (\beta r(t) - y(t)) + K_i \int_{0}^{t} (r(t) - y(t)) \, dt$. With $\beta=0.1$, we have $U(s) = -K_p Y(s) + K_i \frac{R(s)-Y(s)}{s}$. Using $Y(s) = G(s)U(s)$, the closed-loop transfer function is:
  \begin{equation*}
    \frac{Y(s)}{R(s)} = \frac{G(s) K_i / s}{1 + G(s) (K_p + K_i / s)} = \frac{K_i G(s)}{s + G(s) (K_p s + K_i)}
  \end{equation*}
  Let $G(s) = \frac{K}{\tau s + 1}$, so:
  \begin{equation*}
    \frac{Y(s)}{R(s)} = \frac{\frac{K K_i}{\tau}}{s^2 + \frac{K K_p + 1}{\tau} s + \frac{K K_i}{\tau}} =
    \frac{\omega_n^2}{s^2 + 2\zeta \omega_n s + \omega_n^2}
  \end{equation*}
  Zero steady-state error is met since:
  \begin{equation*}
    \lim_{s \to 0} \frac{Y(s)}{R(s)} = 1 \implies e_{ss} = \lim_{t \to \infty} e(t) = \lim_{t \to \infty} y(t) - r(t) = 0
  \end{equation*}
  For overshoot:
  \begin{equation*}
    \text{Overshoot} = e^{-\frac{\pi \zeta}{\sqrt{1-\zeta^2}}} \leq 0.1 \implies \zeta \geq \sqrt{\frac{\ln(0.1)^2}{\ln(0.1)^2 + \pi^2}} \approx 0.591
  \end{equation*}
  Choosing $\zeta = 0.65$ satisfies $\zeta \geq 0.591$. For settling time:
  \begin{equation*}
    T_s = \frac{4}{\zeta \omega_n} \leq 10 \implies 2 \zeta \omega_n \geq \frac{8}{10} \implies \frac{K K_p + 1}{\tau} \geq 0.8
  \end{equation*}
  Using $G(s) = \frac{0.25}{s + 0.2} = \frac{1.25}{5s + 1}$, we have $K = 1.25$, $\tau = 5$. Thus:
  \begin{equation*}
    K_p = 3.4 \geq \frac{\frac{8 \tau}{\beta} - 1}{K} = 2.4
  \end{equation*}
  Also, $\omega_n^2 \geq \frac{16}{\beta^2 \zeta^2} \implies \frac{K K_i}{\tau} \geq \frac{16}{\beta^2 \zeta^2}$, so:
  \begin{equation*}
    K_i = 1.6 \geq \frac{16}{K \zeta^2 \beta^2} \approx 1.51
  \end{equation*}
  In MATLAB and Simulink, all criteria are met.
  \begin{figure}[h!]
    \centering
    \includegraphics[width=0.8\linewidth]{Screenshot 2025-07-08 112241.png}
    \caption{Simulink output for PI controller performance.}
    \label{fig:pi_performance}
  \end{figure}
\end{enumerate}

\section{Controller Design \& Actuator Dynamics}
Consider the system with transfer function $G(s)=\frac{1.5}{0.5s+1}$. Design a controller to satisfy:
\begin{itemize}
  \item Zero steady-state error.
  \item 0\% overshoot.
  \item Settling time less than 0.15 seconds.
\end{itemize}
\begin{enumerate}
  \item What type of controller would you recommend to best satisfy the design criteria? \\
  Since zero steady-state error requires an integrator, a PI controller is appropriate, as P, PD, lag, or lead controllers lack the $\frac{1}{s}$ term. We proceed with a PI controller.

  \item Design an appropriate controller and assess its performance for an input step change from 50 to 60. Provide a detailed explanation of your design process.
  \begin{equation*}
    G(s) = \frac{1.5}{0.5s+1} \implies K=1.5, \tau = 0.5
  \end{equation*}
  The PI controller is $u(t) = K_p (\beta r(t) - y(t)) + K_i \int_0^t (r(t) - y(t)) \, dt$. With $\beta=0.15$, the Laplace transform gives:
  \begin{equation*}
    U(s) = -K_p Y(s) + \frac{K_i (R(s)-Y(s))}{s}
  \end{equation*}
  Using $Y(s) = G(s) U(s)$, the closed-loop transfer function is:
  \begin{equation*}
    \frac{Y(s)}{R(s)} = \frac{G(s) K_i / s}{1 + G(s) (K_p + K_i / s)} = \frac{\frac{K K_i}{\tau}}{s^2 + \frac{K K_p + 1}{\tau} s + \frac{K K_i}{\tau}}
  \end{equation*}
  Zero steady-state error is satisfied since:
  \begin{equation*}
    \lim_{s \to 0} \frac{Y(s)}{R(s)} = 1 \implies e_{ss} = 0
  \end{equation*}
  Zero overshoot requires $\zeta \geq 1$. For settling time, $T_s = \frac{4}{\zeta \omega_n} \leq 0.15 \implies 2\zeta \omega_n \geq \frac{8}{0.15} \approx 53.33$, so:
  \begin{equation*}
    \frac{K K_p + 1}{\tau} \geq 53.33 \implies K_p \geq \frac{\frac{8 \tau}{0.15} - 1}{K} \approx 17.1
  \end{equation*}
  Choosing $K_p = 100$ satisfies this. For $\zeta \geq 1$, we need:
  \begin{equation*}
    \left(\frac{K K_p + 1}{\tau}\right)^2 \geq 4 \frac{K K_i}{\tau} \implies K_i \leq \frac{(K K_p + 1)^2}{4 K \tau} \approx 13467
  \end{equation*}
  Choosing $K_i = 8000$ satisfies this. The Simulink result:
  \begin{figure}[h!]
    \centering
    \includegraphics[width=0.8\linewidth]{Screenshot 2025-07-08 120106.png}
    \caption{Simulink output for PI controller performance (step change from 50 to 60).}
    \label{fig:pi_step_response}
  \end{figure}
  The .mlx file confirms all criteria are met.

  \item Now consider the block diagram where the actuator dynamics are given by $A(s)=\frac{0.99}{0.1s+a}$. Implement your designed controller in this new open-loop structure and evaluate its performance for $a \in \{1,2,5,10\}$.
  \begin{figure}[h!]
    \centering
    \begin{subfigure}[b]{0.48\textwidth}
      \centering
      \includegraphics[width=\linewidth]{Screenshot 2025-07-08 120558.png}
      \caption{Response for $a=1$ (unstable).}
      \label{fig:actuator_a1}
    \end{subfigure}
    \hfill
    \begin{subfigure}[b]{0.48\textwidth}
      \centering
      \includegraphics[width=\linewidth]{Screenshot 2025-07-08 120610.png}
      \caption{Response for $a=2$ (unstable).}
      \label{fig:actuator_a2}
    \end{subfigure}

    \vspace{0.5em}

    \begin{subfigure}[b]{0.48\textwidth}
      \centering
      \includegraphics[width=\linewidth]{Screenshot 2025-07-08 120936.png}
      \caption{Response for $a=5$ (underdamped).}
      \label{fig:actuator_a5}
    \end{subfigure}
    \hfill
    \begin{subfigure}[b]{0.48\textwidth}
      \centering
      \includegraphics[width=\linewidth]{Screenshot 2025-07-08 121022.png}
      \caption{Response for $a=10$ (underdamped).}
      \label{fig:actuator_a10}
    \end{subfigure}
    \caption{System responses for different values of $a$ in actuator dynamics $A(s)=\frac{0.99}{0.1s+a}$.}
    \label{fig:actuator_responses}
  \end{figure}
  \begin{enumerate}
    \item Case $a=1$: Unstable, as shown in Figure~\ref{fig:actuator_a1}.
    \item Case $a=2$: Unstable, as shown in Figure~\ref{fig:actuator_a2}.
    \item Case $a=5$: Underdamped, as shown in Figure~\ref{fig:actuator_a5}.
    \item Case $a=10$: Underdamped, as shown in Figure~\ref{fig:actuator_a10}.
  \end{enumerate}
  For $a \in \{1,2\}$, the time constant $\tau = 0.1/a$ is large, making the actuator slow, causing closed-loop poles to lie in the right half-plane, leading to instability. For $a \in \{5,10\}$, $\tau$ is smaller, making the actuator faster, with poles in the left half-plane, resulting in stability. As $a$ increases, $\zeta$ increases, approaching critical or overdamped behavior.
  \begin{figure}[h!]
    \centering
    \includegraphics[width=0.8\linewidth]{Screenshot 2025-07-08 124404.png}
    \caption{System response with and without actuator dynamics.}
    \label{fig:actuator_comparison}
  \end{figure}
\end{enumerate}

\section*{Conclusion}
This lab demonstrated system identification and control design for a nonlinear system. By linearizing the system around an operating point, deriving a first-order model, and comparing it with the actual system, we validated the linear approximation. The PI controller for the first system met the requirements of zero steady-state error, overshoot less than 10\%, and settling time less than 10 seconds. For the second system, the PI controller achieved zero steady-state error, zero overshoot, and settling time less than 0.15 seconds, with performance varying based on actuator dynamics, as verified through Simulink simulations.

\end{document}




% \documentclass{article}
% \usepackage{graphicx, subfig, fancyhdr, amsmath, amssymb, amsthm, url, geometry, listings, xcolor}
% \usepackage[utf8]{inputenc}
% \usepackage{graphicx}
% \usepackage{subcaption}
% \usepackage{amsmath}

% \usepackage[margin=1in]{geometry}
% % Fix Unicode character issues
% \DeclareUnicodeCharacter{2212}{-}
% % Code listing style
% \lstset{
%   language=C,
%   basicstyle=\ttfamily\small,
%   numbers=left,
%   numberstyle=\tiny,
%   stepnumber=1,
%   frame=single,
%   backgroundcolor=\color{gray!10},
%   keywordstyle=\color{blue}\bfseries,
%   commentstyle=\color{green!50!black},
%   stringstyle=\color{red},
%   breaklines=true,
%   breakatwhitespace=true,
%   showstringspaces=false,
%   tabsize=4
% }
% % Define authors
% \newcommand{\FirstAuthor}{Mohammad Parsa Dini - Std ID: 400101204}
% % \newcommand{\SecondAuthor}{[Second Author Name - Std ID: XXX]} % Placeholder
% \newcommand{\exerciseset}{Control LAB - HW2}
% % Fancy header and footer
% \fancypagestyle{plain}{\fancyhf{}\fancyfoot[RO,LE]{\sffamily\bfseries\thepage}}
% \pagestyle{fancy}
% \fancyhf{}
% \fancyhead[RO,LE]{\sffamily\bfseries\large Sharif University of Technology}
% \fancyhead[LO,RE]{\sffamily\bfseries\large EE 25-403: Control LAB}
% \fancyfoot[LO,RE]{\sffamily\bfseries\large Control LAB HW2}
% \fancyfoot[RO,LE]{\sffamily\bfseries\thepage}
% \renewcommand{\headrulewidth}{1pt}
% \renewcommand{\footrulewidth}{1pt}
% \graphicspath{{figures/}}
% \title{
%   \vspace{-2em}
%   \includegraphics[width=3cm]{logo.png} \\
%   \vspace{0.5em}
%   Control LAB \\
%   \exerciseset
% }
% \author{\FirstAuthor \and \SecondAuthor}
% \date{May 2025}
% % Load hyperref last
% \usepackage{hyperref}

% \begin{document}
% \maketitle
% %-----------------------------------------------------------------------------------------%-----------------------------------------------------------------------------------------
% \section{System Identification and Control}
% \begin{enumerate}
%   \item Linearize the system around the operating point $(u_0,y_0)\approx(0.2,0.5)$.
%   \begin{equation*}
%     \dot{y} = -\frac{y}{5} + \frac{4.5}{20}\sqrt{u} = -0.2 y + 0.225 \sqrt{u}
%   \end{equation*}
%   Since by Taylor representation we have $\sqrt{u}|_{u_0=0.2} \approx \sqrt{u_0} + \frac{u-u_0}{2\sqrt{u_0}} \approx 0.447 + 1.118(u-0.2) = 1.118u + 0.224$. Thus around $u=u_0=0.2$:
%   \begin{equation*}
%     \dot{y} = -0.2y + 0.225(1.118u + 0.224) = -0.2y + 0.252u + 0.0504
%   \end{equation*}
%   which suggests around the operating point: $\dot{y} \approx 0.002$. Taking the Laplace transform of $\dot{y} \approx -0.2y + 0.25u$ results in:
%   \begin{equation*}
%     sY(s) = -0.2Y(s) + 0.25 U(s) \implies H(s) = \frac{Y(s)}{U(s)} = \frac{0.25}{s + 0.2}
%   \end{equation*}
% %-----------------------------------------------------------------------------------------
%   \item Attempt to directly derive a first-order linear model for the nonlinear system shown in the figure. In order to do so, apply a small step change around the operating point $(u_0,y_0)=(0.2,0.5)$ and determine the system's gain and time constant. \\
%   \begin{equation*}
%     \dot{y} = -\frac{y}{5} + \frac{4.5}{20}\sqrt{u} \implies \dot{y} = -0.1 + \frac{4.5}{20}\sqrt{0.2} \approx 0.0006
%   \end{equation*}
%   Let $u_1 = u_0 + \Delta u = 0.2 + 0.02 = 0.22 \implies 0.0006 = -0.2y_1 + \frac{4.5}{20}\sqrt{0.22} \approx 0.500115$, which suggests $\Delta y = 0.000115 \implies \text{gain} = \frac{\Delta y}{\Delta u} = \frac{0.000115}{0.02} \approx 0.00575$. The Simulink result is depicted below:
%   \begin{figure}[h!]
%     \centering
%     \includegraphics[width=0.8\linewidth]{Screenshot 2025-07-07 111042.png}
%     \caption{Simulink result for step response around operating point.}
%     \label{fig:fig1}
%   \end{figure}
%   We can deduce that $K = \frac{0.562-0.5}{0.05} \approx 1.24$ (from Simulink). However, the $K=1.08$ from the .mlx file is more reliable.
%   \begin{figure}[h!]
%     \centering
%     \includegraphics[width=0.8\linewidth]{Screenshot 2025-07-07 112245.png}
%     \caption{Simulink output for gain calculation.}
%     \label{fig:fig2}
%   \end{figure}
%   $y_2 = 0.5 + 0.63 (0.562 - 0.5) \approx 0.539 \implies t_2 = 34.7 \implies \tau = t_2 - 30 = 4.7$ (from Simulink).
%   \begin{figure}[h!]
%     \centering
%     \includegraphics[width=0.8\linewidth]{Screenshot 2025-07-07 112713.png}
%     \caption{Simulink output for time constant calculation.}
%     \label{fig:fig3}
%   \end{figure}
% %-----------------------------------------------------------------------------------------
%   \item Compare the linear model derived in parts 1 and 2 with the actual system by applying a step change around the operating point $(u_0,y_0)=(0.2,0.5)$. Plot the output of models and the real systems on the same axes. \\
%   The results are shown in the .mlx file attached to the GitHub repository and this report. The result from Simulink:
%   \begin{figure}[h!]
%     \centering
%     \includegraphics[width=0.8\linewidth]{Screenshot 2025-07-07 113152.png}
%     \caption{Comparison of linear model and actual system response.}
%     \label{fig:fig4}
%   \end{figure}
%   As you can see, $G_a(s)$ performed better in MATLAB and Simulink.
% %-----------------------------------------------------------------------------------------
%   \item Design a PI controller to raise the output from 0.5 to 0.6 while satisfying the following requirements: \\
%   - Zero steady-state error in response to a unit step input. \\
%   - Overshoot less than 10\%. \\
%   - Settling time less than 10 seconds. \\
%   The PI controller is given by $u(t) = K_p (\beta r(t) - y(t)) + K_i \int_{0}^{t} (r(t) - y(t)) \, dt$. Since $\beta=0$, we have $U(s) = -K_p Y(s) + K_i \frac{R(s)-Y(s)}{s}$. Using $Y(s) = G(s)U(s)$, the closed-loop transfer function is:
%   \begin{equation*}
%     \frac{Y(s)}{R(s)} = \frac{G(s) K_i / s}{1 + G(s) (K_p + K_i / s)} = \frac{K_i G(s)}{s + G(s) (K_p s + K_i)}
%   \end{equation*}
%   Let $G(s) = \frac{K}{\tau s + 1}$, so:
%   \begin{equation*}
%     \frac{Y(s)}{R(s)} = \frac{\frac{K K_i}{\tau}}{s^2 + \frac{K K_p + 1}{\tau} s + \frac{K K_i}{\tau}} =
%     \frac{\omega_n^2}{s^2 + 2\zeta \omega_n s + \omega_n^2}
%   \end{equation*}
%   The first criterion (zero steady-state error) is met since:
%   \begin{equation*}
%     \lim_{s \to 0} \frac{Y(s)}{R(s)} = 1 \implies e_{ss} = \lim_{t \to \infty} e(t) = \lim_{t \to \infty} y(t) - r(t) = 0
%   \end{equation*}
%   For the second criterion (overshoot):
%   \begin{equation*}
%     \text{Overshoot} = e^{-\frac{\pi \zeta}{ \sqrt{1-\zeta^2}}} \leq 0.1 \implies \zeta \geq \sqrt{\frac{\ln(0.1)^2}{\ln(0.1)^2 + \pi^2}} \approx 0.591
%   \end{equation*}
%   Choosing $\zeta = 0.65$ satisfies $\zeta \geq 0.591$. For the third criterion (settling time):
%   \begin{equation*}
%     T_s = \frac{4}{\zeta \omega_n} \leq 10 \implies 2 \zeta \omega_n \geq \frac{8}{10} \implies \frac{K K_p + 1}{\tau} \geq 0.8
%   \end{equation*}
%   Using $G(s) = \frac{0.25}{s + 0.2} = \frac{1.25}{5s + 1}$, we have $K = 1.25$, $\tau = 5$ and $\beta=0.1$. Thus:
%   \begin{equation*}
%     K_p =3.4 \geq \frac{\frac{8 \tau}{\beta} - 1}{K} = 2.4
%   \end{equation*}
%   Choosing $K_p = 3.4$ satisfies the condition.
  
%   also $\omega_n^2 \geq \frac{16}{\beta^2 \zeta^2} \xrightarrow[]{} \frac{KK_i}{\tau} \geq \frac{16}{\beta^2 \zeta^2}$ must hold which suggests:
%   \begin{equation*}
%       K_i = 1.6 \geq \frac{16}{K\zeta^2\beta^2} = 1.51
%   \end{equation*}
%   which clearly the last criterion holds as well.
%   In matlab and simulink, all criteria are met.
%   \begin{figure}[h!]
%     \centering
%     \includegraphics[width=0.8\linewidth]{Screenshot 2025-07-08 112241.png}
%     %\caption{Comparison of linear model and actual system response.}
%     \label{fig:fig5}
%   \end{figure}
% \end{enumerate}

% \section{Controller Design & Actuator Dynamics}
% Consider the system described by the following transfer function: $G(s)=\frac{1.5}{0.5s+1}$. Our aim is to design a controller which satisfies the following requirements:\\
%     - zero steady state error \\
%     - 0\% overshoot \\
%     - settling time less than 0.15 seconds \\
% \begin{enumerate}
%     \item what type of controller would you recommend to best satisfy the design criteria?
%     \\
%     Since, one criterion is zeros steady state, we need some controller with $\frac{1}{s}$ integrator; thus, P, PD, lag ,lead and etc. would not be useful. PI and PID look appropriate. We continue with PI. \\
    
%      % -----------------------------------------------------------------------------------------
%     \item Design an appropriate controller and assess its performance for an input step change from 50 to 60. Provide a detailed explanations of your design process. \\
    
%     \begin{equation*}
%         G(s) = \frac{1.5}{0.5s+1} \xrightarrow[]{} K=1.5 , \tau = 0.5
%     \end{equation*}
    
%     \begin{equation*}
%         u(t) = K_p (\beta r(t) - y(t)) + K_i \int_0^\infty (r(t) - y(t)) dt
%     \end{equation*}
%     Getting laplace transform implies that($\beta=0$):
%     \begin{equation*}
%         U(s) = -K_p Y(s) + \frac{K_i(R(s)-Y(s))}{s}
%     \end{equation*}
%     Now let $Y(s) = G(s) U(s)$, therefore we get:
%     \begin{equation*}
%         G(s) G_c(s) = \frac{Y(s)}{R(s)} = \frac{K_i s^{-1}}{G(s)^{-1} + K_i s^{-1} + K_D}
%     \end{equation*}
%     Since $G(s) = \frac{K}{\tau s +1 }$ and $G_{cl}(s) = \frac{K_i s^{-1}}{\frac{1+\tau s}{K} + K_i s^{-1} + K_p} = \frac{ \frac{KK_i}{\tau} }{s^2 + \frac{KK_p+1}{\tau}s + \frac{KK_i}{\tau}} $, thus:
%     \begin{equation*}
%         \lim_{s \to 0} \frac{Y(s)}{R(s)} = 1 \implies e_{ss} = \lim_{t \to \infty} e(t) = \lim_{t \to \infty} y(t) - r(t) = 0
%     \end{equation*}
%     Furthermore, the zero overshoot implies $\zeta \geq 1$. Now we move on to the third criteria $T_s = \frac{4}{\zeta \omega_n} \leq \beta$ and $2\zeta \omega_n \geq \frac{8}{\beta} \xrightarrow[]{} \frac{KK_p+1}{\tau} \geq \frac{8}{\beta}$. Thus:
%     \begin{equation*}
%         K_p \geq \frac{-1 + \frac{8\tau}{\beta}}{K}
%     \end{equation*}
%     For $G(s)$, we derived $G(s) = \frac{0.25}{s+0.2} = \frac{1.25}{5s+1}$. Thus, $K=1.5$, $\tau=0.5$ and $\beta=0.15$.
%     \begin{equation*}
%         K_p = 100 \geq \frac{-1 + \frac{8\tau}{\beta}}{K} \approx 17.1
%     \end{equation*}
%     which clearly holds. Now we   move on to the next criterion. Since $\zeta \geq 1 \xrightarrow[]{} 2\zeta \omega_n = \frac{KK_p+1}{\tau} \geq 2_\omega_n$ which suggests:
%     \begin{equation*}
%         (\frac{KK_p+1}{\tau})^2 \geq 4_\omega_n^2 \xrightarrow[]{} (\frac{KK_p+1}{\tau})^2 \geq 4\frac{KK_i}{\tau}
%     \end{equation*}
%     which results in:
%     \begin{equation*}
%         K_i =200\leq \frac{\tau}{4K}(\frac{KK_p+1}{\tau})^2 = \frac{(KK_p+1)^2}{4K\tau} = 261.3
%     \end{equation*}
%     which also holds. Now Choosiing $K_p =200$ will give $K_i \leq13467 $, so we choose $K_i=8000$.\\
%     And here is the simulink resutls:
    
%     \begin{figure}[h!]
%     \centering
%     \includegraphics[width=0.8\linewidth]{Screenshot 2025-07-08 120106.png} 
%     %\caption{Comparison of linear model and actual system response.}
%     \label{fig:fig7}
%   \end{figure}
%     Also done in matlab .mlx file. As you can see again, all the criteria are met.\\
    
    
    
    
    
    
    
   
    
    
%     % -----------------------------------------------------------------------------------------
%     \item Now consider the block diagram, where the actuator dynamics are given by $A(s)=\frac{0.99}{0.1s+a}$. Implement your designed controller in this new open-loop structure and evaluate its performance for the values of $a \in \{1,2,5,10\}$.
%     \\
    
    
    
    
    
    
%     \begin{figure}[h!]
%     \centering
%     \begin{subfigure}[b]{0.45\textwidth}
%       \centering
%       \includegraphics[width=0.4\linewidth]{Screenshot 2025-07-08 120558.png}
%       \caption{Response for $a=1$ (unstable).}
%       \label{fig:fig5}
%     \end{subfigure}
%     \hfill
%     \begin{subfigure}[b]{0.45\textwidth}
%       \centering
%       \includegraphics[width=0.4\linewidth]{Screenshot 2025-07-08 120610.png}
%       \caption{Response for $a=2$ (unstable).}
%       \label{fig:fig6}
%     \end{subfigure}

%     \vspace{0.5em}

%     \begin{subfigure}[b]{0.45\textwidth}
%       \centering
%       \includegraphics[width=0.4\linewidth]{Screenshot 2025-07-08 120936.png}
%       \caption{Response for $a=5$ (underdamped).}
%       \label{fig:fig7}
%     \end{subfigure}
%     \hfill
%     \begin{subfigure}[b]{0.45\textwidth}
%       \centering
%       \includegraphics[width=0.4\linewidth]{Screenshot 2025-07-08 121022.png}
%       \caption{Response for $a=10$ (underdamped).}
%       \label{fig:fig8}
%     \end{subfigure}
%     \caption{System responses for different values of $a$ in actuator dynamics $A(s)=\frac{0.99}{0.1s+a}$.}
%     \label{fig:actuator_responses}
%   \end{figure}

%   \begin{enumerate}
%     \item Case $a=1$: Unstable, as shown in Figure~\ref{fig:fig5}.
%     \item Case $a=2$: Still unstable, as shown in Figure~\ref{fig:fig6}.
%     \item Case $a=5$: Underdamped, as shown in Figure~\ref{fig:fig7}.
%     \item Case $a=10$: Underdamped, as shown in Figure~\ref{fig:fig8}.
%   \end{enumerate}
    
%     While $a \in \{1,2\}$, $\tau $ is big and the actuator is slow and thus the closed-loop poles lie on the right side of the $j\omega $ axis, and hence the system will be unstable as we saw above. \\
%     While $a \in \{5,10\}$, $\tau $ is small and the actuator is fast and thus the closed-loop poles lie on the left side of the $j\omega $ axis, and hence the system will be stable as we saw above. With $a$ rising, $\zeta$ will increase as well, and the system will get critically damped and absolutely damped by big $a$'s.
    
    
    
%     % \begin{enumerate} \item case $a=1 \xrightarrow[]{} \text{Unstable}$ \begin{figure}[h!] \centering \includegraphics[width=0.8\linewidth]{Screenshot 2025-07-08 120558.png} 
%     % %\caption{Comparison of linear model and actual system response.} 
%     % \label{fig:fig8} 
%     % \end{figure} 
%     % \item case $a=2 \xrightarrow[]{} still \text{unstable}$ 
%     % \begin{figure}[h!] 
%     % \centering 
%     % \includegraphics[width=0.8\linewidth]{Screenshot 2025-07-08 120610.png} 
%     % %\caption{Comparison of linear model and actual system response.} 
%     % \label{fig:fig9}
%     % \end{figure} 
%     % \item case $a=5 \xrightarrow[]{} \text{underdamped}$ 
%     % \begin{figure}[h!] 
%     % \centering 
%     % \includegraphics[width=0.8\linewidth]{Screenshot 2025-07-08 120610.png}
%     % %\caption{Comparison of linear model and actual system response.} 
%     % \label{fig:fig10} 
%     % \end{figure}
%     % \item case $a=10 \xrightarrow[]{} \text{underdamped}$ 
%     % \begin{figure}[h!] 
%     % \centering 
%     % \includegraphics[width=0.8\linewidth]{Screenshot 2025-07-08 121022.png}
%     % %\caption{Comparison of linear model and actual system response.} 
%     % \label{fig:fig10} 
%     % \end{figure}
%     % \end{enumerate}
%     And here is the System Control with & with out the Actuation:
%     \begin{figure}[h!] 
%     \centering 
%     \includegraphics[width=0.8\linewidth]{Screenshot 2025-07-08 124404.png}
%     %\caption{Comparison of linear model and actual system response.} 
%     \label{fig:fig17} 
%     \end{figure}

% \end{enumerate}

% \section*{Conclusion}
% This lab demonstrated the process of system identification and control design for a nonlinear system. By linearizing the system around an operating point, deriving a first-order model, and comparing it with the actual system, we validated the linear approximation. The PI controller design successfully met the requirements of zero steady-state error, overshoot less than 10\%, and settling time less than 10 seconds, as verified through Simulink simulations.

% \end{document}
