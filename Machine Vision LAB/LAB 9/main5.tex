\documentclass{article}
\usepackage{amsmath}
\usepackage{listings}
\usepackage{color}
\usepackage{graphicx}

\begin{document}

\title{Bonus Question for Lab 9:  \\ Building an AND Gate Using a Perceptron}
\author{Mohammad Parsa Dini \\ Student ID: 400101204}
\date{}
\maketitle


\section*{Introduction}
A perceptron is a simple neural network model that can be used to implement logical gates such as the AND gate. In this document, we will build an AND gate using a perceptron and visualize it with some test cases using Python.

\section*{Perceptron Model}
The perceptron model for an AND gate is defined with the following parameters:
- Inputs: \(x_1\) and \(x_2\)
- Weights: \(w_1 = 1\), \(w_2 = 1\)
- Bias: \(b = -1.5\)

The output of the perceptron is given by:


\[
\text{output} = 
\begin{cases} 
1 & \text{if} \, (w_1 \cdot x_1 + w_2 \cdot x_2 + b) \geq 0 \\
0 & \text{otherwise}
\end{cases}
\]



\section*{Python Code for Visualization}
The following Python code implements the perceptron model for the AND gate and visualizes the results for all possible input combinations.

\begin{lstlisting}[language=Python, caption=Perceptron Implementation for AND Gate]
import numpy as np
import matplotlib.pyplot as plt

def perceptron(x1, x2, w1=1, w2=1, b=-1.5):
    weighted_sum = w1 * x1 + w2 * x2 + b
    return 1 if weighted_sum >= 0 else 0

# Test the perceptron with all input combinations for an AND gate
inputs = [(0, 0), (0, 1), (1, 0), (1, 1)]
outputs = [perceptron(x1, x2) for x1, x2 in inputs]

# Print the results
for inp, out in zip(inputs, outputs):
    print(f"Input: {inp} -> Output: {out}")

# Visualize the perceptron decision boundary
plt.figure()
for x1, x2 in inputs:
    plt.scatter(x1, x2, color='blue' if perceptron(x1, x2) else 'red')
plt.xlabel('x1')
plt.ylabel('x2')
plt.title('AND Gate with Perceptron')
plt.grid(True)
plt.show()
\end{lstlisting}

\section*{Results}
The perceptron correctly implements the AND gate, as shown by the following output:
\begin{verbatim}
Input: (0, 0) -> Output: 0
Input: (0, 1) -> Output: 0
Input: (1, 0) -> Output: 0
Input: (1, 1) -> Output: 1
\end{verbatim}

The decision boundary and visualization are displayed in the plot, where blue points represent output 1 and red points represent output 0.

\begin{figure}[h!]
    \centering
    \includegraphics[width=0.7\linewidth]{Figure_2.png} % Adjust the width as needed
    \caption{The output of the model for AND gate and the decision boundary.}
    \label{fig:figure1}
\end{figure}

\end{document}
